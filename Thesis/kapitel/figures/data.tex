\begin{figure}
    \center
    \begin{tikzpicture}
        \Vertex[x=0.5, y=0.5,color=white,opacity=0.5,label=p, size=1, shape=circle]{p}
        \Vertex[x=-1,y=3,color=white,opacity=0.5,label=q, size=1, shape=circle]{q}
        \Vertex[x=2,y=2.5,color=white,opacity=0.5,label=r, size=1, shape=circle]{r}
        \Vertex[x=4,y=4,color=white,opacity=0.5,label=s, size=1, shape=circle]{s}
        \Vertex[x=2.5,y=6.5,color=white,opacity=0.5,label=t, size=1, shape=circle]{t}
        \Vertex[x=5.5,y=1.75,color=white,opacity=0.5,label=u, size=1, shape=circle]{u}
        \Vertex[x=7.5,y=4,color=white,opacity=0.5,label=v, size=1, shape=circle]{v}
        \Vertex[x=7.5,y=6.5,color=white,opacity=0.5,label=w, size=1, shape=circle]{w}
        \Edge[](q)(r)
        \Edge[](p)(r)
        \Edge[](r)(s)
        \Edge[](w)(s)
        \Edge[](u)(s)
        \Edge[](t)(s)
        \Edge[](v)(s)
        \end{tikzpicture}
    \caption[Independence Structure of a Graph G with 8 Nodes]{Graph G=(V,E) with m=8 random variables from a dataset $\mathcal{D}$ with eight features and samples $\vect{x} = (x_p, x_q, x_r, x_s, x_t, x_u, x_v, x_w) \in \mathbb{R}^8$. The conditional independency structure is given by the edges of the graph. In this case the graph is a tree, but dependency structures are generally not limited to trees. }
    \label{fig:data}
\end{figure}