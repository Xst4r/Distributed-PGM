%Tikz Magic
\tikzstyle{data}=[rectangle split,
                  rectangle split parts=3,
                  rectangle split part fill={tugray!30,tugreen!20, tuorange!20},
                  rounded corners,
                  draw=black, 
                  very thick,
                  text centered]

\tikzstyle{coordinator}=[rectangle split,
                        rectangle split parts=2,
                        rectangle split part fill={tugray!30,tugreen!20},
                        rounded corners,
                        draw=black, 
                        very thick,
                        text centered]
\begin{figure}[h!]
    \begin{subfigure}[t]{.5\textwidth}
    \center
    \begin{tikzpicture}[node distance=2cm]
        \node [data] (a)  {CPU \nodepart{second} Memory \nodepart{third} Data};
        \node [data]  at (4,4)  (b)    {CPU \nodepart{second} Memory \nodepart{third} Data};
        \node [data]  at (0,4)  (c)    {CPU \nodepart{second} Memory \nodepart{third} Data};
        \node [data]  at (4,0)  (d)    {CPU \nodepart{second} Memory \nodepart{third} Data};
        \Edge[color=black](a)(b)
        \Edge[color=black](a)(c)
        \Edge[color=black](b)(c)
        \Edge[color=black](a)(d)
        \Edge[color=black](b)(d)
        \Edge[color=black](c)(d)
        \end{tikzpicture}
        \subcaption[Cap]{Fully connected distributed learners each with the usual hardware stack and locally observed data.}
    \end{subfigure}
    \begin{subfigure}[t]{.5\textwidth}
        \center
        \begin{tikzpicture}[node distance=2cm]
            \node [data] (a)  {CPU \nodepart{second} Memory \nodepart{third} Data};
            \node [data]  at (4,4)  (b)    {CPU \nodepart{second} Memory \nodepart{third} Data};
            \node [data]  at (0,4)  (c)    {CPU \nodepart{second} Memory \nodepart{third} Data};
            \node [data]  at (4,0)  (d)    {CPU \nodepart{second} Memory \nodepart{third} Data};
            \node [coordinator] at (2,2) (e) {CPU \nodepart{second} Memory};
            \Edge[color=black](a)(e)
            \Edge[color=black](b)(e)
            \Edge[color=black](c)(e)
            \Edge[color=black](d)(e)
            \end{tikzpicture}
            \subcaption[Cap]{Distributed learners in a federated system. Each learner is connected only to the central coordinator, which has no acess to the data.}
        \end{subfigure}
    \caption[Distributed vs. Federated Architecture]{Distributed vs. Federated network architecture, featuring four devices either connected between each other or connected to a coordinator device. The coordinator in the right illustration does not have access to the data.}
    \label{fig:architecutre}
\end{figure}