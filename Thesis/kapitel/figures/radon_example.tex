\begin{figure}[H]
    \center
    \begin{subfigure}[t]{.45\textwidth}
        \center 
        \begin{tikzpicture}
            \clip (-1,-1) rectangle (5,5);
            \Vertex[x=0,y=0,color=tugreen,opacity=0.5,label=1, size=0.5, shape=diamond]{a}
            \Vertex[x=3,y=0,color=tugreen, opacity=0.5,label=1, size=0.5, shape=diamond]{b}
            \Vertex[x=1.5,y=3,color=tugreen, opacity=0.5,label=1, size=0.5, shape=diamond]{c}
            \Vertex[x=1.5,y=1,color=tuorange, opacity=0.5,label=-3, size=0.5, shape=circle]{d}
            \Edge(a)(b)
            \Edge(b)(c)
            \Edge(c)(a)
            \end{tikzpicture}
            \subcaption[]{Four points with a single, unique, solution to \eq~\ref{eq:radonopt}}
            \label{fig:uniqueradon}
    \end{subfigure}
    \hspace{1cm}
    \begin{subfigure}[t]{.45\textwidth}
        \center
        \begin{tikzpicture}
            \clip (-1,-1) rectangle (5,5);
            \Vertex[x=0,y=0,color=tugreen,opacity=0.5,label=1, size=0.5, shape=diamond]{a}
            \Vertex[x=3,y=0,color=tugreen, opacity=0.5,label=1, size=0.5, shape=diamond]{b}
            \Vertex[x=1.5,y=3,color=tugreen, opacity=0.5,label=1, size=0.5, shape=diamond]{c}
            \Vertex[x=1.5,y=1,color=tuorange, opacity=0.5,label=-3, size=0.5, shape=circle]{d}
            \Vertex[x=0,y=2,color=tuorange, opacity=0.5,label=0, size=0.5, shape=circle]{e}
            \Edge(a)(b)
            \Edge(b)(c)
            \Edge(c)(a)
            \Edge(d)(e)
            \end{tikzpicture}
            \subcaption[]{Five points in $\mathbb{R}^2$ with no unique solution to \eq~\ref{eq:radonopt}}
            \label{fig:overdetradon}
    \end{subfigure}

    \caption[Radon Point]{Geomtric interpretation of radon points in $\mathbb{R}^2$. Both cases yield a solution to the radon point computation with \fig~\ref{fig:overdetradon} having more than one possible solution to the problem (all points on the line between the red circles, that are inside the triangle)}
    \label{fig:radon}
\end{figure}