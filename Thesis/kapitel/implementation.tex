% implementation.tex

\chapter{Implementation}
\label{chapter:ch4}
As our programming language of choice, python was chosen to implement the various aggregation mechanics. 
This allows for fast and efficient execution and evaluation, while still allowing to interface with C and C++  libraries if necessary.

The module is written in pure python and is mainly using well known python modules such as NumPy, SciPy, Pandas and Scikit-Learn.

Additionally, we use the PX Module for probabilistic graphical modes.
PX is a module, which allows data-based structure learning as well as all operations necessary for training and evaluating probabilistic graphical models. 
These include likelihood optimization via first order proximal gradient descent \cite{piatkowski2018exponential}, belief propagation and loopy belief propagation algorithms and direct inference tasks such as MAP-State, Marginal probabilities.
Furthermore the package features an interface for applying functions to the gradient and weights directly during optimization.
These so called hooks allow for direct manipulation of the gradient for e.g. custom regularization functions or simply to implmenets ones own convergence criterium for the gradient descent.

Distributed learning is implemented as a simulation between two classes. The Coordinator and the Models, while in a real-world setting we have actual distributed devices, there is no difference between receiving a number of parameter vectors from distributed devices or in a simulated fashion from a list of simulated devices.
While one could argue that communication between devices may not be as stable, we can simply omit the parameters we do not receive, since they are not known anyhow.
Depending on the aggregation method the Coordinator has access to different statistics, such as number of models, parameter vectors and objective function value. 
However, the local data or the average sufficient statistics are never shared with the coordinator node.

Moreover, in a real-world setting the devices may terminate optimization at different times and may send the parameters asynchronous. 
We assume the process to be synchronous and wait for all local devices to terminate.
Usually, we can rely on a running aggregate if we receive parameters asynchronously or we wait until a certain number of parameters have been sent to the coordinator to proceed.



\section{Structure}

\begin{figure}[!htb]
    \center
    \begin{tikzpicture}
        \clip (0,-0.5) rectangle (16,12);
        \Vertex[x=8,y=5,color=tuorange,opacity=0.5,label=Coordinator, size=2, shape=rectangle]{c}
        \Vertex[x=2,y=2,color=tugreen, opacity=0.5,label=$\mathcal{M}^1$, size=1, shape=circle]{d1}
        \Vertex[x=6,y=2,color=tugreen, opacity=0.5,label=$\mathcal{M}^2$, size=1, shape=circle]{d2}
        \Vertex[x=14,y=2,color=tugreen, opacity=0.5,label=$\mathcal{M}^k$, size=1, shape=circle]{dn}
        \Vertex[x=2,y=0,color=white, opacity=0, size=0, shape=diamond]{data1}
        \Vertex[x=6,y=0,color=white, opacity=0, size=0, shape=diamond]{data2}
        \Vertex[x=14,y=0,color=white, opacity=0, size=0, shape=diamond]{datan}
        \Vertex[x=5,y=9,color=tugray,opacity=0.5,label=Aggregate, size=1.7, shape=diamond]{a}
        \Edge[Direct ,bend=-10, label=$\vect{\theta}^{1}$](d1)(c)
        \Edge[Direct ,bend=-10, label=$\vect{\theta}^{2}$](d2)(c)
        \Edge[Direct ,bend=-10, label=$\vect{\theta}^{n}$](dn)(c)
        \Edge[Direct ,bend=-10, label=$[\vect{\theta}]$](c)(a)
        \Edge[Direct,bend=-10](c)(d1)
        \Edge[Direct,bend=-10](c)(d2)
        \Edge[Direct,bend=-10](c)(dn)
        \Edge[Direct, label=$\mathcal{D}_1$](data1)(d1)
        \Edge[Direct, label=$\mathcal{D}_2$](data2)(d2)
        \Edge[Direct, label=$\mathcal{D}_n$](datan)(dn)
        \Edge[Direct ,bend=-10, label=$\hat{\vect{\theta}}$](a)(c)
        \end{tikzpicture}
    \caption[Implementation Chart]{Implementation of the continuous structure, which allows for communication between coordinator and all devices. The coordinator then computes the aggregate and may send feedback in form of the aggregate back to all devices. The devices themselves may be connected as well. First the local data $\mathcal{D}_i$ is collected on each device. Then each device obtains the local parameters $\vect{\theta}^i$, which it then sends to the coordinator. Once a sufficient number of parameter vectors have arrived the coordinator computes an aggregate and optionally provides feedback about the quality, e.g., the aggregate.}
    \label{fig:dist}
\end{figure}
\fig~\todo{REF} illustrates the structure of the module. 
The central coordinator node receives parameters from distributed devices(models) $\mathcal{M}$, which are then aggregated by any instance of the aggregation module. 
This makes it further possible to propagate additional messages such as terminations or updates via the coordinator.
In practice, that is, for testing the coordinator distributes the data in a preprocessing step.

We simulate the data generation by shuffling and equally distributing the data set onto each device, such that the chunks are a partition of the original data.
Each round we add more data to the local data set, i.e., while the local data set consists of $\abs{D}_i = n$ samples we only add a certain subsample $m \leq n$ to the training. This increases with each round to simulate the data collection on each device. 
Results are reported for each round.


\section{Model Training}
Training is straightforward with PX. 
Each device keeps its own instance of a PX Model, cotaining the number of samples, the sufficient statistics, objectives and weights.
We only have to keep in mind that the PX python module is a wrapper around the actual c++ module.
Hence, we have to treat all arrrays and data as C-Contiguous, i.e., in C Memory order.
The data is discretized in a previously set number of quantiles, e.g., 10 quantiles and then treates as an unsigned 16-bit integer to further reduce memory requirements.
\section{Model Aggregation}
Model Aggregation is done in python with numpy,scipy to improve runtimes of matrix-vector operations.
\section{Evaluation}
For evaluation we use accuracy and f1 score either computed in pure python or supported bei the sklearn machine learning library.