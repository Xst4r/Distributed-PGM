% Model_Aggregation.tex

\chapter{Model Aggregation}
\label{chapter:ch3}

\begin{definition}[parbox=false]{Model Aggregation}
    Let $\mathcal{H}$ be a space of all feasible solutions for some parametrized machine learning task with given structure and $\mathcal{F}$ be the space of all possible functions $f: \mathcal{H}^k \rightarrow \mathcal{H}$ that map a set of feasible solutions to a single solution.
    Furthermore, let $\vect{M}$ be a random variable and $\mathcal{M} = (m^1, \ldots, m^k) \in \mathcal{H}^k$ be some collection of $k$ models sampled from $\vect{M}$.

    Our goal is to find an aggregate function $f \in \mathcal{F}$ such that it minimizes some loss function with a limited amount of samples from $\vect{M}$:
    \begin{equation}
        \label{eq:regret}
        \arg\min_{f \in \mathcal{F}} d(f(\mathcal{\mathcal{M}}), m^*),
    \end{equation}
    where $m^*$ is the optimal aggregate model and $d: \mathcal{H} \times \mathcal{H} \rightarrow \mathbb{R}$ is some quality measure or distance function.
\end{definition}

In the context of this work feasible solutions are from a parametrized family of denisites, i.e., the solution space $\mathcal{H} = \theta$
$\mathcal{H} = \mathbb{R}^d$, where $d$ is the number of parameters of the model and $\vect{\theta} \in \mathbb{R}^d$.
The aggregate function is then $f: \mathbb{R}^{d \times k} \rightarrow \mathbb{R}^{d}$ such that the input is a matrix with $d \times k$ entries, where each column corresponds to a single parameter vector.

For exponential families we already know from section \ref{ssec:asymp} that $\vect{M} \sim \mathcal{N}(\vect{m}^*, i(\vect{m}^*)/n)$ is normally distributed.

Now, let algorithm~\ref{alg:magg} be a generalized version of the model aggregation on distributed systems. 
Each round, we update the local data set and continue updating the model parameters in parallel. 
We then send the intermediate results to a coordinator node to check for convergence.
Note, that depeneding on the stopping criterion, we may need to transmit the number of observed samples as well. 
The stopping criterion may also involve aggregating the intermediate result or computing some distance function $d: \mathbb{R}^d \times \mathbb{R}^d  \rightarrow \mathbb{R}$ in order to determine whether the training should be stopped.
\begin{algo}{Distributed Model Aggregation}
    \begin{algorithm}[H]
    \caption{Distributed Learning}
        \begin{algorithmic}[1]
            \label{alg:magg}
            \REQUIRE Datasets $\mathcal{\vect{D}} = \{\mathcal{D}_1, \ldots,\: \mathcal{D}_k\}$, Graph $G=(V,\:E)$, States $s \in \mathbb{R}^{\abs{V}}$ and Stopping Criterion C \\
            \ENSURE Aggregate Model $\vect{\tilde{\theta}}$  \\
            \WHILE{not STOP} \STATE{%
            \COMMENT{Train models on each device}\\
            \FORALL{Devices $i = 1, \ldots, k$} 
            \STATE{
                $\mathcal{D}_i \leftarrow \text{update}(\mathcal{D}_i)$ \\
                $\vect{\theta}^{(i)} \leftarrow \text{train}(\mathcal{D}_i,\: G,\: s)$\\
                $\text{send}(\vect{\theta}^{(i)})$\\
            } 
            \ENDFOR
            }\\
            \COMMENT{On the Coordinator} \\
            \IF{C is fulfilled}
                \STATE{Broadcast STOP} \\
            \ENDIF
            \ENDWHILE
            \FORALL{Devices $i = 1, \ldots, k$}
            \STATE{$\text{send}(\vect{\theta}^{(i)})$\\}
            \ENDFOR
            \STATE{$\vect{\tilde{\theta}} \leftarrow \text{aggregate}((\vect{\theta}^{(1)},\ldots,\: \vect{\theta}^{(k)}))$}\\
            %\IF{<condition>} \STATE {<text>} \ELSE \STATE{<text>} \ENDIF
            %\IF{<condition>} \STATE {<text>} \ELSIF{<condition>} \STATE{<text>} \ENDIF
            %\FOR{<condition>} \STATE {<text>} \ENDFOR
            %\FOR{<condition> \TO <condition> } \STATE {<text>} \ENDFOR
            %\REPEAT \STATE{<text>} \UNTIL{<condition>}
            %\LOOP \STATE{<text>} \ENDLOOP
            %\AND, \OR, \XOR, \NOT, \TO, \TRUE, \FALSE
            \RETURN {$\vect{\tilde{\theta}}$}
        \end{algorithmic}
    \end{algorithm}
\end{algo}

Let us now discuss aggregation function $f \in \mathcal{F}$.

\section{Aggregation Mechanisms}
\subsection{Averaging}

The most intuitive approach to model aggregation is done by simply computing the unbiased average of all model parameters.
\begin{equation}
    \frac{1}{n} \sum_{i=1}^n \theta^i
\end{equation}
Recall that in section \ref{ssec:asymp} we showed that the maximum likelihood estimator for canonical exponential families is asymptotically normal.
Hence, each $\vect{\theta}^i$ is one sample from the distribution $\vect{\theta}^i \dot{\sim} \mathcal{N}(\vect{\theta}^*, i(\theta)^{-1})$.
While we have no direct knowledge of $\vect{\theta^*}$, we can use the maximum likelihood estimators for mean and covariance for normal distributions which are

\begin{tcolorbox}
    Given a data set $\mathcal{D}$ with an underlying multivariate normal distribution and let $\vect{x}^i \in \mathcal{D}$ be a single sample from $\mathcal{D}$.
    The maximum likelihood estimators for mean and covariance of a multivariate normal distribution are
    \begin{equation}
        \begin{split}
            &\vect{\mu}_{MLE} = \frac{1}{n} \sum_{i=1}^n \vect{x}^i \\
        &\vect{\Sigma}_{MLE} = \frac{1}{n} \sum_{i=1}^n (\vect{x}^i - \vect{\mu}) \cdot  (\vect{x}^i - \vect{\mu})^T.
        \end{split}
    \end{equation}
\end{tcolorbox}
For $\vect{\theta}^i$ we then have $\vect{\mu}_{MLE} = \frac{1}{n} \sum_{i=1}^n \vect{\theta}^i$, which is exactly the average of all local model parameters.
Thus the arithmetic mean maximizes the likelihood.
\begin{proof}{Maximum Likelihood Estimator Normal Distribution}
    For $\mathcal{N}(\vect{\theta}^*, i(\vect{\theta})^{-1}/n)$ the we want to find the maximizer for $\vect{\theta}^*$ for the maximum likelihood estimator. 
    The partial derivative of $\ell$ \wrt $\vect{\theta}^*$ given $\vect{\theta}$ is:
    \begin{equation}
        \begin{split}
        \frac{\partial}{\partial \vect{\theta}^*}&\ell_{\mathcal{N}}(\vect{\theta}^*, \vect{i(\vect{\theta}^*)^{-1}/n}; \mathcal{D}) =  \frac{\partial}{\partial \vect{\theta}^*} \sum_{\vect{\theta} \in \mathcal{D}} \log \mathcal{N}(\vect{\theta}; \vect{\theta}^*, \vect{i(\vect{\theta}^*)^{-1}/n})\\
        &=  \sum_{\vect{\theta} \in \mathcal{D}} \frac{\partial}{\partial \vect{\theta}^*} \log \frac{1}{\sqrt{2 \pi} \abs{i(\vect{\theta}^*)^{-1}/n}} \cdot \exp\bigg(\frac{1}{2} (\vect{\theta} - \vect{\theta}^*)^Ti(\vect{\theta}^*)/n(\vect{\theta} - \vect{\theta}^*) \bigg) \\
        &=  \sum_{\vect{\theta} \in \mathcal{D}} \frac{\partial}{\partial \vect{\theta}^*} \log (\frac{1}{\sqrt{2 \pi} \abs{i(\vect{\theta}^*)^{-1}/n}}) + \frac{1}{2} (\vect{\theta} - \vect{\theta}^*)^Ti(\vect{\theta}^*)/n(\vect{\theta} - \vect{\theta}^*) \\
        &=  \sum_{\vect{\theta} \in \mathcal{D}} \frac{\partial}{\partial \vect{\theta}^*} \frac{1}{2} (\vect{\theta} - \vect{\theta}^*)^Ti(\vect{\theta}^*)/n(\vect{\theta} - \vect{\theta}^*) - \log \sqrt{2 \pi} \abs{i(\vect{\theta}^*)^{-1}/n} \\
        &= - \frac{1}{2} i(\vect{\theta}^*)/n \sum_{\vect{\theta} \in \mathcal{D}} (\vect{\theta} - \vect{\theta}^*) = 0 \\
        &\Leftrightarrow \vect{\theta}^* = \frac{1}{\abs{\mathcal{D}}} \sum_{\vect{\theta} \in \mathcal{D}}  \vect{\theta}
        \end{split}
    \end{equation}
    Note, that $i(\vect{\theta}^*)$ is defined \wrt the original probability distribution $\mathbb{P}_{\vect{\theta}}$ and can be treated in the same way $\vect{\Sigma}$ does in the standard proof.
\end{proof}



\subsection{Weighted Averaging}
Usually we do not have a sufficient amount of samples to perform simple model averaging with expected behaviour close to the asymptotic case. 
When the sample size is small we possibly observe models that in reality have a very low probability of being observed. 
The importance of these models are then overestimated since all local models contribute equally to the model average.
With the weighted arithmetic mean, we introduce  weights $\vect{\lambda}$ assigning some measure of importance to a local model, with larger weights being more important and smaller ones being less important.
Additionally we may normalize the weights such that $\sum_{i=1}^k \lambda_i = 1, \lambda_i \geq 0$, i.e., the elements sum to one, which represents as a discrete probability distribution over $\vect{\theta}^i$, where $\mathbb{P}(\vect{X} = \vect{\theta}^i) = \lambda_i$ for a finite set $\mathcal{M} = \{\vect{\theta}^1, \ldots, \vect{\theta}^k\}$ of parameter samples from a random variable $\vect{M}$.

Then, instead of computing the arithmetic mean, which asummes all weights to be equal to $1$, we compute
\begin{equation}
     \hat{\vect{\theta}} = \sum_{i=1}^k \lambda_i \theta^i = \tilde{\mathbb{E}}_{\mathcal{M}}[\vect{M}],
\end{equation}
which is the empirical expected value of $\vect{M}$ if and only if $\vect{\lambda}$ is a probability measure.
Obtaining these weights is a challenging task as this requires some prior knowledge about each local model.
We may consider evaluating each model using a performance metric or scoring function and use the normalized scores as means to obtain $\vect{\lambda}$
\begin{equation}
    \lambda_i = \frac{\log f(\vect{\theta}^i)}{\sum_{j=1}^k \log f(\vect{\theta}^j)},
\end{equation}
where $f: \mathbb{R}^d \rightarrow \mathbb{R}_+$ is an arbitrary positive real valued function, that maps each local model to some score, probability or likelihood.

Now let $f(\vect{\theta}^i) = \log \mathcal{N}(\vect{\theta}^i \lvert \ldots)$ be the likelihood function of a normal distribution.
Each weight is then given by
\begin{equation}
    \lambda_i = \frac{\log \frac{1}{\sqrt{2\pi\lvert \vect{\Sigma} \rvert}} \cdot \exp^{-\frac{1}{2} \cdot (\vect{\theta}^i - \vect{\mu})^T\vect{\Sigma}^{-1}(\vect{\theta}^i - \vect{\mu})}}{\sum_{j=1}^k \log \frac{1}{\sqrt{2\pi\lvert \vect{\Sigma} \rvert}} \cdot \exp^{-\frac{1}{2} \cdot (\vect{\theta^j} - \vect{\mu})^T\vect{\Sigma}^{-1}(\vect{\theta^j} - \vect{\mu})}},
\end{equation}
which are just the normalized weights based on all known local models.
Intuitively, this measures the likelihood of each model based on the predetermined distribution.
If the actual distribution $\vect{\theta}$ follows is know, this represents the distance between the true parameters $\vect{\theta}^*$ and some local sample.

However, if the actual distribution is known we can simply use the MAP-Estimate instead of the Maximum Likelihood Method as this alllows us to include prior information about the parameter distribtion.

\paragraph*{MAP Estimator}
When dealing with likelihood optimization we usually assume the prior probability to be constant, that is, we treat all possible parameters as equally likely.
The Maximum A-Posteriori(MAP)-Estimator includes the prior information such that we assume the parameters to follow some probability distribution.
This is especially useful in settings, where data is scarce.
However, if the underlying assumption of the prior's distribution does not hold they dominate the estimator resulting in worse models.
Recall that we obtain the posterior probability for a set of local models $\mathcal{M}$ as the product of likelihood and prior
\begin{equation}
    \begin{split}
    \arg\max_{\vect{\theta}}\mathbb{P}(\vect{\theta} \lvert \mathcal{M} ) &= \arg\max_{\vect{\theta}} \frac{\mathbb{P}(\mathcal{M}\lvert \vect{\theta}) \mathbb{P}(\vect{\theta}) }{\mathbb{P}(\mathcal{M})} \\
    &\propto  \arg\max_{\vect{\theta}}\mathbb{P}(\mathcal{M}\lvert \vect{\theta}) \mathbb{P}(\vect{\theta})  ,
    \end{split}
\end{equation}
where the denonimator does not change the optimum \wrt $\vect{\theta}$ and can therefore be omitted.

In Section~\ref{ssec:asymp} we have shown that the maximum likelihood estimator is asymptotically normal.
Hence, if the true distribution of $\vect{\theta}$ is know we can use this as prior information to compute the MAP-estimate as weighted average of the prior information and the MLE-Estimate for the mean:
\begin{equation}
    \label{eq:mapest}
    \begin{split}
    \arg\max_{\vect{\theta}} \ell(\vect{\theta}; \mathcal{M}, \vect{\lambda}) &= \arg\max_{\vect{\theta}} \sum_{i=1}^k \log \mathbb{P}(\mathcal{M}\lvert \vect{\theta}) \mathbb{P}(\vect{\theta}) \\
    &= \arg\max_{\vect{\theta}} \sum_{i=1}^k \log \mathcal{N}(\vect{\theta}^i ; \vect{\theta}, \Sigma) + \log \mathcal{N}(\vect{\theta}; \vect{\theta}^*, i(\vect{\theta}^{-1}/n)).
    \end{split}
\end{equation}
Taking the derivative of \eq~\ref{eq:mapest} \wrt $\vect{\theta}$ we get
\begin{equation}
    \vect{\theta} = \frac{\vect{\theta}^*  + n \cdot \Sigma \cdot i(\vect{\theta}^*) \sum_{i=1}^k \vect{\theta}^i}{1 + n \cdot \Sigma \cdot i(\vect{\theta}^*) \cdot k},
\end{equation}
which is the weighted average of the maximum likelihood estimator for $\vect{\theta}$ and the mode of the distribution $\vect{\theta}$ follows weighted by the variance.

In practice we often encounter the usage of the bayesian average~\cite{de2011bayesian}, which is the weighted average between the prior mean and the sample mean
\begin{equation}
    \vect{\theta} = \frac{\tau \vect{\theta}^* + \sum_{i=1}^k \vect{\theta}^i}{\tau + k},
\end{equation}
where $\tau$ is a hyperparameter that controls the importance of the prior mean, i.e., we add the prior mean $\tau$ times to the average. 

\paragraph*{Performance-Weighted-Averaging}

Instead of using a federated approach we can also rely on a federated-decentralized hybrid approach, establishing connections or using existing connections between local devices to enable communication.
We can model connections and devices as a network graph:
\begin{definition}{Network}
    Let $\mathcal{N} = (V, E)$ be a network graph with indexed nodes $V_i, \: i \in I$ and edges $E$, where the nodes are local devices hold a local model $m^{(i)}$ each.
    Edges between two nodes respresent a connection between these two devices, i.e., the devices can communicate with each other.
\end{definition}

Then, let $\vect{\theta}^{i}$ be the model parameters of i$^{th}$ model $m^{(i)}$ with locally collected data $\mathcal{D}^{(i)}$.
Transmitting the model parameters $\vect{\theta}^{i}$  at node $V_i$  to other connected devices  $V_j$ in peer-to-peer fashion allows us to effectively evaluate the generalization capabilites of a model.
We obtain weights for the i$^{th}$ model based on the average performance on all of its neighbors, by computing the running average and variance of some performance measure such as accuracy or likelihood.

Depending on the type of connection it may be required to use an online algorithm to compute average and variance based on scoring individual parameters from connected devices.
On such algorithm is Welford's algorithm, which can be used to compute running average and variance for some scoring function.
Welford's algorithm has the property to asynchronously update the running average, while only retaining the number of elements that have already been averaged and the average itself.

The communication process is shown in \fig \ref{fig:dist}, where on the left, node $i$ sends its local parameters to neighbouring nodes. The neighbours $k$ and $j$ return the local score, determined by some scoring function $S: \mathcal{H} \times \mathbb{R}^n \rightarrow \mathbb{R}$ with local model $\mathcal{M} \in \mathcal{H}$ and parameters $\vect{\theta} \in \mathbb{R}^d$.

\begin{figure}[!htb]
    \begin{subfigure}{.5\textwidth}
    \center
    \begin{tikzpicture}
        \clip (-1,-1) rectangle (6.5,6.5);
        \Vertex[x=0,y=0,color=red,opacity=0.5,label=$m^{i}$, size=1]{i}
        \Vertex[x=5,y=5,color=tugreen,opacity=0.5,label=$m^{j}$, size=1]{j}
        \Vertex[x=3.5,y=1,color=tugreen,opacity=0.5,label=$m^{k}$, size=1]{k}
        \Vertex[x=4.5,y=2,color=white,opacity=0.2,label=$m^{l}$, size=1]{l}
        \Vertex[x=1,y=3.5,color=tugreen,opacity=0.5,label=$m^{m}$, size=1]{m}
        \Vertex[x=1.5,y=5.5,color=white,opacity=0.2,label=$m^{n}$, size=1]{n}
        \Edge[Direct ,bend=-8, label=$\vect{\theta}^{i}$](i)(j)
        \Edge[,bend=-8](k)(l)
        \Edge[,bend=-8](j)(l)
        \Edge[,bend=-8](j)(m)
        \Edge[,bend=--8](n)(m)
        \Edge[,bend=-8](j)(n)
        \Edge[Direct,bend=-8, label=$\vect{\theta}^{i}$](i)(k)
        \Edge[Direct,bend=-8, label=$\vect{\theta}^{i}$](i)(m)
        \end{tikzpicture}
        \subcaption[Cap]{Node $i$ sending the local parameter vector to its neighbors.}
    \end{subfigure}
    \begin{subfigure}{.5\textwidth}
        \center
        \begin{tikzpicture}
            \clip (-1,-1) rectangle (6.5,6.5);
            \Vertex[x=0,y=0,color=tugreen,opacity=0.5,label=$m^{i}$, size=1]{i}
            \Vertex[x=5,y=5,color=red,opacity=0.5,label=$m^{j}$, size=1]{j}
            \Vertex[x=3.5,y=1,color=red,opacity=0.5,label=$m^{k}$, size=1]{k}
            \Vertex[x=4.5,y=2,color=white,opacity=0.2,label=$m^{l}$, size=1]{l}
            \Vertex[x=1,y=3.5,color=red,opacity=0.5,label=$m^{m}$, size=1]{m}
            \Vertex[x=1.5,y=5.5,color=white,opacity=0.2,label=$m^{n}$, size=1]{n}
            \Edge[Direct ,bend=-8, label=$S(\vect{\theta}^{i})_k$](j)(i)
            \Edge[,bend=-8](k)(l)
            \Edge[,bend=-8](j)(l)
            \Edge[,bend=-8](j)(m)
            \Edge[,bend=-8](n)(m)
            \Edge[,bend=-8](j)(n)
            \Edge[Direct,bend=-8, label=$S(\vect{\theta}^{i})_j$](k)(i)
            \Edge[Direct,bend=-8, label=$S(\vect{\theta}^{i})_m$](m)(i)
            \end{tikzpicture}
            \subcaption[Cap]{Neighbors of $i$ returning a score based on the message of node $i$ and some scoring metric $S$}
        \end{subfigure}
    \caption[Message passing in Networks.]{Example Network with 6 Nodes. Messages passed between sender(orange) and receiver(green) in a general network graph with neighbors  $\mathcal{N}(i) = \{j,k,m\}$ of vertex $i$. Every neighbor of $i$ receives a message, the model parameters, and  returns a result based on the specified scoring metric back to the sender.}
    \label{fig:network}
\end{figure}

In practice the general assumption is that each device is only connected to two other devices, that is the underlying network structure is a chain graph.
However, similar approaches on general network graphs have been discussed in depth for example by Wolff~\cite{wolff2013local} with thresholding functions.
The communication is not restricted to neighbouring nodes only and once all nodes agree on a value to a certain extent, i.e., all nodes stop sending messages, the algorithm terminates.
In the case of model aggregation one would choose a thresholding function that either computes the pairwise distance between parameters or once the average score is sufficiently good.

Once all local models have received an average score the local parameters along with the score are sent to the coordinator in order to compute the weighted average

\begin{equation}
    \tilde{\vect{\theta}}_{DEC} = \frac{1}{\sum_{i=1}^k \bar{S}_i} \cdot \sum_{i=1}^k \bar{S}_i \vect{\theta}^{(i)},
\end{equation}

where $\bar{S}_i$ is the average score collected from each neighbour of $i$. 
The normalization is applied to ensure that $\sum_{i=1}^k S_i = 1$, i.e., the aggregate is a convex combination of the local models.

\subsection{Radon Machines}

A more robust method for solving the issue of model aggregation is the computation Radon Points, which is a centerpoint of intersecting convex hulls and related to tukey depth and the geometric median. 
Centerpoint methods as well as the geomtric median have a increase robustness when compared to arithmetic means.
Methods, such as the arithmetic mean suffer heavily from outliers, while center point methods do not.
This is reflected in the fact that we may observe "bad" models of up to fifty percent of the sample size, while still being able to obtain the correct center point.

\begin{definition}[parbox=false]{Centerpoint}
    Given a set of points $T \subset \mathbb{R}^{d}$ and a set  $\vect{H} \subset \mathbb{R}^{d-1}$ of separating hyerplanes that partition $T$  into two roughly equal sized sets $T_1, T_2$, a center point $p \in \mathbb{R}^d$ is a point, where all hyperplanes $h \in \vect{H}$ intersect.

    Iff the partition is in general position, i.e., the partition is a unique solution, the radon point is guaranteed to be the center point of $T$, as shown by Peterson.~\cite{peterson1972geometry}
\end{definition}

Both concepts, centerpoint and geometric median, are designed to find the center of mass of a given set of points.
However, while both concepts are closely related, they are different generalizations to the median in high-dimensional space.
While the geometric median is well defined, there exists no algorithm to exactly compute the it in $\mathbb{R}^n$ dimensions.
Instead, Kamp et al. \cite{kamp2017effective} introduced Radon Machines as method for model ensembling to compute a centerpoint for a set of models.
A radon point can be found in polynomial time \wrt the number of dimensions of the space the points are located in.
\paragraph*{Radon Points}
Given a pair of convex hulls defined by two sets of points definined the hull $S_1, S_2$, a radon point is a point that is contained within the intersection of the two convex hulls of $S_1, S_2$. 
Radon's Theorem was originally proposed by Radon in 1921 \cite{radon1921mengen}:

\begin{threm}[label=thm:radon]{Radon's Theorem}
    Given a set of of points in euclidean space
    \begin{equation}
        S = \{\vect{x}_1, \ldots \vect{x}_{d+2}\} \subset \mathbb{R}^d,
    \end{equation}
   there exists a partition of S into two subsets $S_1, S_2$ such that the intersection of the convex hulls spanned by both sets is not empty, that is
    
    \begin{equation}
        \exists S_1, S_2, \;\; S_1 \cup S_2 = S, \; S_1 \cap S_2 = \emptyset: Conv(S_1) \cap Conv(S_2) \neq \emptyset,
    \end{equation}

    where $Conv(\cdot)$ is the Convex Hull for a given set.
    Any point contained  in the intersection between the two convex hulls is a radon point.
\end{threm}

Example~\ref{ex:radon} illustrates Radon's Theorem in 2-dimensional euclidean space.
Given a set of $d+2 =4$ points we can express the radon point as intersection of the convex hulls.
In this case the convex hulls are formed by a triangle around the fourth point.
The convex hulls intersect exactly on the point in the center of the triangle and it is the unique solution to this problem.

\begin{example}{Radon Points in $\mathbb{R}^2$}
    \label{ex:radon}
    Consider this example of Radon's Theorem in $\mathbb{R}^2$ on the left with four points $\{(0,0), (3,0), (1.5, 3), (1.5,1)\}$, where the former three form a equilateral triangle and the latter is the center point of that triangle.
    The solution to the linear equations in terms Radon's Theorem is $\{1,1,1\},\{-3\}$ in the same as  the points. 
    The example on the right contains an additional point outside of the triangle, which results in an overdetermined system of linear equations, where the solution is no longer unique.
    
    \begin{figure}[H]
    \center
    \begin{subfigure}[t]{.45\textwidth}
        \center 
        \begin{tikzpicture}
            \clip (-1,-1) rectangle (5,5);
            \Vertex[x=0,y=0,color=tugreen,opacity=0.5,label=1, size=0.5, shape=diamond]{a}
            \Vertex[x=3,y=0,color=tugreen, opacity=0.5,label=1, size=0.5, shape=diamond]{b}
            \Vertex[x=1.5,y=3,color=tugreen, opacity=0.5,label=1, size=0.5, shape=diamond]{c}
            \Vertex[x=1.5,y=1,color=tuorange, opacity=0.5,label=-3, size=0.5, shape=circle]{d}
            \Edge(a)(b)
            \Edge(b)(c)
            \Edge(c)(a)
            \end{tikzpicture}
            \subcaption[]{Four points with a single, unique, solution to \eq~\ref{eq:radonopt}}
            \label{fig:uniqueradon}
    \end{subfigure}
    \hspace{1cm}
    \begin{subfigure}[t]{.45\textwidth}
        \center
        \begin{tikzpicture}
            \clip (-1,-1) rectangle (5,5);
            \Vertex[x=0,y=0,color=tugreen,opacity=0.5,label=1, size=0.5, shape=diamond]{a}
            \Vertex[x=3,y=0,color=tugreen, opacity=0.5,label=1, size=0.5, shape=diamond]{b}
            \Vertex[x=1.5,y=3,color=tugreen, opacity=0.5,label=1, size=0.5, shape=diamond]{c}
            \Vertex[x=1.5,y=1,color=tuorange, opacity=0.5,label=-3, size=0.5, shape=circle]{d}
            \Vertex[x=0,y=2,color=tuorange, opacity=0.5,label=0, size=0.5, shape=circle]{e}
            \Edge(a)(b)
            \Edge(b)(c)
            \Edge(c)(a)
            \Edge(d)(e)
            \end{tikzpicture}
            \subcaption[]{Five points in $\mathbb{R}^2$ with no unique solution to \eq~\ref{eq:radonopt}}
            \label{fig:overdetradon}
    \end{subfigure}

    \caption[Radon Point]{Geomtric interpretation of radon points in $\mathbb{R}^2$. Both cases yield a solution to the radon point computation with \fig~\ref{fig:overdetradon} having more than one possible solution to the problem (all points on the line between the red circles, that are inside the triangle)}
    \label{fig:radon}
\end{figure} 
    The normalizer is then the sum of the weights of either partition
    \begin{equation}
        \begin{split}
            A &= \sum_{i\in I} a_i = - \sum_{j \in J} a_j \\
            &= 1 + 1 + 1 = -(-3) = 3 \\
        \end{split}
    \end{equation}
    and plugging that into \eq~\ref{eq:radonpoint} we obtain the intersection of the convex hulls as
    \begin{equation}
        \begin{split}
            \bar{\vect{\theta}}_{RAD} &= \sum_{i\in I} \frac{a_i}{A} \vect{\theta}^i = - \sum_{j \in J} \frac{a_j}{A} \vect{\theta}^j \\
            &= \frac{1}{3} \vect{\theta}^{I_1} + \frac{1}{3} \vect{\theta}^{I_2} + \frac{1}{3} \vect{\theta}^{I_3} = 1 \vect{\theta}^{J_1} \\
            & =  (1.5, 1)^T = (1.5, 1)^T.
        \end{split}
    \end{equation}
    Note that there are two different cases in $\mathbb{R}^2$, the one shown above and the second one where four points form two intersecting lines. Geometrically, the intersection is then the center point.

\end{example}
   

\paragraph*{Computation of Radon Points}

The problem of finding such as point can be solved by formulating a system of linear equations, where the radon point is the convex combination all elements from either partition $S_1, S_2$.
Now let $S = \{\vect{\theta}^{(1)}, \ldots \vect{\theta}^{(d+2)}\} \subset \mathbb{R}^d$, be a set consisting of $d+2$ local model parameter vectors.
We can find the model aggregate in terms of the radon point by using the solution obtained from the following linear program:

\begin{equation}
    \label{eq:radonopt}
    \begin{split}
        \min_{\vect{\lambda}} \quad &1 \\
        s.t. \;\; &\sum_{i=1}^{d+2} \lambda_i \vect{\theta}^{(i)} = \vect{0}\\
             \;\; &\sum_{i=1}^{d+2} \lambda_i = 0 \\
             \; \; & \lambda_1 = 1.
    \end{split}
\end{equation}

Having $d+2$ $\mathbb{R}^d$-dimensional points results in an overdetermined system with no unique solution.
This issue can be solved by adding an additional constraint, e.g. fixing one arbritary variable to a constant. 
Note, that we still need a total of $d+2$ parameter vectors to compute the radon point.
This allows, in absence of linear dependencies, finding a unique solution to the problem.

However, any solution to the system of linear equations presented in \eq~\ref{eq:radonopt}is a radon point and there may exist multiple solutions.
One particular issue that we may run into is that the maximum likelihood estimators obtained from sampling the same random variable approch each other in the limit of $n \rightarrow \infty$, possibly leading to the introduction of linear dependencies.
While this is an issue for the sytem of linear equation, as the solution may not be unique anymore, we can simply use the optimization problem to minimize the sum of squares to find any one solution.
This means however, that we lose some the property of the radon point being the only center point as shown bei Peterson et al.~\cite{peterson1972geometry}
If the solution to the optimization problem is unique, the partition is unique and thus the radon point is the center point of all model parameter vectors.
Hence, if there is no unique solution there is not much information about the position of the radon point except that it is some sort of center point and contained in the intersection of the two convex hulls.
The least squares solution is obtained by minimizing
\begin{equation}
    \label{eq:lstsq}
        \min_{\vect{\lambda}} \norm{S\vect{\lambda} - b}^2_2,
\end{equation}
where $b=(1, 0, \ldots, 0) \in \mathbb{R}^{d+2}$ is a zero vector with a single one at the first position and $S$ is the coefficient matrix contaning model parameters (cf. \eq~\ref{eq:radonopt}).

For a valid solution $\vect{\lambda}$ and Sets $I,J$, such that $I$ contains all positive entries of $\vect{\lambda}$ and $J$ contains negative or zero entries of $\vect{\lambda}$ the unique solution and model aggregate $\vect{\bar{\theta}}_{RAD}$ is then given by:

\begin{equation}
    \label{eq:radonpoint}
    \begin{split}
    A &= \sum_{i\in I} \lambda_i = - \sum_{j\in J} \lambda_j, \; \;\lambda_i > 0, \; \lambda_j \leq 0 \\
    \vect{\bar{\theta}}_{RAD} &= \sum_{i\in I} \frac{\lambda_i}{A} \vect{\theta}^{(i)} = - \sum_{j\in J} \frac{\lambda_j}{A} \vect{\theta}^{(j)}.
    \end{split}
\end{equation}
The radon point $\vect{\bar{\theta}}_{RAD}$ is a point where the convex combination of both index sets $I, J$ coincide, that is, the partition of $S$ is given by the set of positive and negative entries of $\vect{\lambda}$.

As $\sum_{i=1}^{2+d} \lambda_i = 0$ it must be that the sum of positive entries in $\vect{lambda}$ equals the negative entries and therefore $A$ is the normalizer for both partition sets, which ensures that $\vect{\bar{\theta}}_{RAD}$ is a convex combination of both sets.

\begin{algo}{Radon Machine}
    \begin{algorithm}[H]
    \caption[Radon Machine for parallelized model aggregation for exponential family models]{Radon Machine for parallelized model aggregation for exponential family models. Each iteration consists of 1) Obtaining the coefficients for the convex combination of parameters and 2) computing the center point as the convex combination. In the last step we only have a single set of $r$ parameter vectors that has to be aggregated to obtain the final result. In case of linear dependencies between parameter vectors we use the least squares optimization instead.}
        \begin{algorithmic}[1]
            \label{alg:radon}
            \REQUIRE Model Parameters $\mathcal{M} = \{\vect{\theta}^1, \ldots,\: \vect{\theta}^{r^{h}}\}$, Radon Number $r$, Aggregation Steps $h$ \\
            \ENSURE Aggregate Model $\vect{\tilde{\theta}}_{RAD}$  \\
            \COMMENT{Partition Parameters into $r^{h-1}$ Subsets of Size $r$}
            \STATE{$\mathcal{M}^{\text{old}} \leftarrow \{\{\vect{\theta}^1, \ldots \vect{\theta}^r\}_1, \ldots, \{\vect{\theta}^1, \ldots \vect{\theta}^r\}_{r^{h-1}}\}$\\}
            \FOR{$i = h-1, \ldots, 0$}
            \IF{i==0}
                \IF{Matrix induced by $\mathcal{M}^{old}_1$ is singular}
                \STATE{$\vect{\lambda} \leftarrow$ solveLeastSquares($\mathcal{M}^{old}_1$)(\eq~\ref{eq:radonopt})}
                \ELSE
                \STATE{$\vect{\lambda} \leftarrow$ solveLinearEquations($\mathcal{M}^{old}_1$)(\eq~\ref{eq:radonopt})}
                \ENDIF
                \STATE{$\mathcal{M}^{new} \leftarrow aggregate(\mathcal{M}^{old}_1, \vect{\lambda})$; \\
                Break;\\}
            \ENDIF
            \STATE{
            \COMMENT{Aggregate Models in Parallel}\\
            \STATE{$\mathcal{M}^{\text{new}}\leftarrow \emptyset$ }
            \FOR{$j=1, \ldots, r^i$} 
            \IF{Matrix induced by $\mathcal{M}^{old}_j$ is singular}
            \STATE{$\vect{\lambda} \leftarrow$ solveLeastSquares($\mathcal{M}^{old}_j$)(\eq~\ref{eq:radonopt})}
            \ELSE
            \STATE{$\vect{\lambda} \leftarrow$ solveLinearEquations($\mathcal{M}^{old}_j$)(\eq~\ref{eq:radonopt})}
            \ENDIF
            \STATE{
                $\vect{\theta}_{RAD}^j \leftarrow aggregate(\mathcal{M}^{old}_j, \vect{\lambda})$; (\eq~\ref{eq:radonpoint})\\
                $\mathcal{M}^{\text{new}}\leftarrow \mathcal{M}^{\text{new}} \cup \vect{\theta}_{RAD}^j$;  \\
            } 
            \ENDFOR
            }\\
            \COMMENT{Partition $\mathcal{M}^{new}$ into $r^{i-1}$ subsets of size $r$}\\
            \STATE{$\mathcal{M}^{\text{old}} \leftarrow  \{\{\vect{\theta}^1, \ldots \vect{\theta}^r\}_1, \ldots, \{\vect{\theta}^1, \ldots \vect{\theta}^r\}_{r^{i-1}}\}$\\}
            \ENDFOR
            \RETURN {$\mathcal{M}^{new} = \{\vect{\theta}_{RAD}\}$}
        \end{algorithmic}
    \end{algorithm}
\end{algo}

\paragraph*{Limitations}
Given a set of parameter vectors we can compute their center point as radon point in polynomial time \wrt the dimensionality of the model parameters.
However, this leads to certain drawbacks as number of models required to iteratively compute the radon points scales exponentially with the number of features of the data.
For the aggregation of exponential family model this implies that the number of models scales with the number of parameters of that model as we aggregate a set of model parametes $\vect{\theta}$.

Given a model with 1000 parameters, we would already require more than one million models for $h > 1$, which requires large quantity of models and data.
Otherwise we would limit ourselves to $h=1$, which still requires $d+2 = 1002$ models for a single aggregation.

We alleviate this issue by sampling additional model parameters from a distribution centered around the local model parameters. 
Recall that in section \ref{ssec:asymp} we showed, that the MLE for exponential family 
models is asymptotically normal.
If we assume the local model parameters to be the "true" model parameters \wrt to the data seen we can sample additional model parameters. 
This can be done for each local model up until the required amount of parameters $(d+2)^h$ have been sampled.


\subsection{Bootstrap Aggregagation}

With generative modelling we can easily sample additional data from local models by transferring the local model parameters to the coordinator node.
Sampling can be realized in various different ways such as Gibbs Sampling or Perturb and Map.
We then create a new Dataset from our samples and use this to train a new global model. 
Instead of directly aggregating $\vect{\theta}^k$ we first sample new data  $\hat{\vect{x}}^k \sim \mathbb{P}_{\vect{\theta}^k}$ and then train new models on the aquired sample data.
However, for exponential families training $k$ models with average sufficient statistics $\vect{\mu}^k$ is equal to training a single model with the average of all sufficient statistics:
\begin{equation}
    \begin{split}
        \bar{\vect{\theta}}_{BS} &= \arg\min_{\vect{\theta}} \frac{1}{k}\sum_{i=1}^k \frac{1}{\abs{\mathcal{D}}_i} \sum_{\hat{\vect{x}}^k \in \mathcal{D}_i} \log f(\vect{\theta}; \hat{\vect{x}}^k) \\
        &=  \arg\min_{\vect{\theta}} \frac{1}{k}\sum_{i=1}^k \frac{1}{\abs{\mathcal{D}}_i} \sum_{\hat{\vect{x}}^k \in \mathcal{D}_i} \inner{\phi(\hat{\vect{x}}^k)}{\vect{\theta}} - A(\vect{\theta}) \\
        &= \arg\min_{\vect{\theta}} \frac{1}{k}\sum_{i=1}^k \inner{\frac{1}{\abs{\mathcal{D}}_i} \sum_{\hat{\vect{x}}^k \in \mathcal{D}_i} \phi(\hat{\vect{x}}^k)}{\vect{\theta}} - A(\vect{\theta}) \\
        &= \arg\min_{\vect{\theta}} \frac{1}{k}\sum_{i=1}^k \inner{\vect{\mu}^k}{\vect{\theta}} - A(\vect{\theta}) \\
        & =  \arg\min_{\vect{\theta}} \inner{\frac{1}{k}\sum_{i=1}^k  \vect{\mu}^k}{\vect{\theta}} - A(\vect{\theta})
    \end{split}
\end{equation}

Alternatively, we can consecutively sample new parameter vectors first
Recall that according to \ref{ssec:asymp} the MLE is asymptotically normal, then we have $\hat{\theta}^{k_i}  \sim \mathcal{N}(\vect{\theta}^k, i(\vect{\theta}^k)^{-1}/n)$, assuming that each local parameters are the ground truth.
Followed by using the MAP estimate for each $\vect{x}^{k_i} = \arg\max_{\vect{x}} \mathbb{P}_{\hat{\vect{\theta}}^{k_i}}(\vect{x})$ obtaining data sets of MAP estimates from parameter distributed around the k-th local model.


\section{Model Sampling}
In Section~\ref{ssec:asymp} we have shown the asymptotic normality for the maximum likelihood estimator of canonical exponential families.
While the true distribution \wrt $\vect{\theta}^*$ is usually unknown, we may still be able to take advantage of the asymptotic properties. 

Given some local model $\vect{\theta}^i$ obtained from data sampled on the i-th device we may assume any other sample from the same device resulting in a different parameter vector $\vect{\hat{\theta}}^i \sim \mathcal{N}(\vect{\theta}^i, \cdot)$ to be distributed around the original sample. 
Since any $\hat{\vect{\theta}}$ sampled from this distribution can be interpreted as the MLE for a different i.i.d. sample we are able to generate new samples based on a single local estimate.

There are two things to consider when sampling new parameters.
First we have to determine where the generate the new samples.
Sampling on the devices itself leads to parallelism and thus speedup, while also increasing communication requirements, as all samples need to be communicated.
Generating the samples on the coordinator node minimizes communication cost, while increasing computation time for the sampling. 
Second, we have to find an efficient way to generate the covariance matrix as computation of the fisher information $i(\vect{\theta}^i)$ may not be feasible in terms of memory requirement especially when sampling on the local devices.

Here we chose to generate additional samples on the coordinator while proposing different methods to obtain covariances.

\paragraph*{Covariance Matrix Generation}


