% header.tex
\documentclass[a4paper,11pt,twoside,english]{book}
\usepackage[a4paper,left=3.5cm,right=2.5cm,bottom=3.5cm,top=3cm]{geometry}
\usepackage{todonotes}
\usepackage[english]{babel}

%usepackage[pdftex]{graphicx}
\usepackage{amsmath,amssymb}


% Theorem-Umgebungen
\usepackage[amsmath,thmmarks]{ntheorem}

% Korrekte Darstellung der Umlaute
\usepackage[utf8]{inputenc}
\usepackage[T1]{fontenc}

% Algorithmen
\usepackage[plain,chapter]{algorithm}
\usepackage{algorithmic}

\usepackage{enumerate}
\usepackage{acronym}
\usepackage{tcolorbox}
\usepackage{csquotes}
\usepackage[hidelinks]{hyperref}
\usepackage[parfill]{parskip}


\usepackage{tikz}
\usetikzlibrary{fit,positioning,arrows,automata}
\usepackage{tikz-network}
\usepackage{caption}
\usepackage{subcaption}
% Bibtex deutsch
%\usepackage{bibgerm}
\usepackage[citestyle=numeric,bibstyle=alphabetic,backend=biber, backref=true]{biblatex}
\addbibresource{literature.bib}
% URLs
\usepackage{url}

% Caption Packet
\usepackage[margin=0pt,font=small,labelfont=bf]{caption}
% Gliederung einstellen
\setcounter{secnumdepth}{5}
\setcounter{tocdepth}{5}

% Theorem-Optionen %
\theoremseparator{.}
\theoremstyle{change}
\newtheorem{theorem}{Theorem}[section]
\newtheorem{satz}[theorem]{Satz}
\newtheorem{lemma}[theorem]{Lemma}
\newtheorem{korollar}[theorem]{Korollar}
\newtheorem{proposition}[theorem]{Proposition}
% Ohne Numerierung
\theoremstyle{nonumberplain}
\renewtheorem{theorem*}{Theorem}
\renewtheorem{satz*}{Satz}
\renewtheorem{lemma*}{Lemma}
\renewtheorem{korollar*}{Korollar}
\renewtheorem{proposition*}{Proposition}
% Definitionen mit \upshape
\theorembodyfont{\upshape}
\theoremstyle{change}
\theoremstyle{nonumberplain}
% Kursive Schrift
\theoremheaderfont{\itshape}
\newtheorem{notation}{Notation}
\newtheorem{konvention}{Konvention}
\newtheorem{bezeichnung}{Bezeichnung}
\theoremsymbol{\ensuremath{\Box}}
\newtheorem{beweis}{Beweis}
\theoremsymbol{}
\theoremstyle{change}
\theoremheaderfont{\bfseries}
\newtheorem{bemerkung}[theorem]{Bemerkung}
\newtheorem{beobachtung}[theorem]{Beobachtung}
\newtheorem{beispiel}[theorem]{Beispiel}
\newtheorem{problem}{Problem}
\theoremstyle{nonumberplain}
\renewtheorem{bemerkung*}{Bemerkung}
\renewtheorem{beispiel*}{Beispiel}
\renewtheorem{problem*}{Problem}

% Algorithmen anpassen %
\renewcommand{\algorithmicrequire}{\textit{Input:}}
\renewcommand{\algorithmicensure}{\textit{Output:}}
\floatname{algorithm}{Algorithm}
\renewcommand{\listalgorithmname}{List of Algorithms}
\renewcommand{\algorithmiccomment}[1]{\textcolor{tuorange}{// #1}}

% Zeilenabstand einstellen %
\renewcommand{\baselinestretch}{1.25}
% Floating-Umgebungen anpassen %
\renewcommand{\topfraction}{0.9}
\renewcommand{\bottomfraction}{0.8}
% Abkuerzungen richtig formatieren %
\usepackage{xspace}
\newcommand{\wrt}{w.r.t.\@\xspace} 
\newcommand{\eg}{e.\nolinebreak[4]\hspace{0.125em}\nolinebreak[4]g.\@\xspace}
\newcommand{\ie}{i.e.\@\xspace}
\newcommand{\dahe}{d.\nolinebreak[4]\hspace{0.125em}h.\nolinebreak[4]\@\xspace}
\newcommand{\etc}{etc.\@\xspace}
\newcommand{\evtl}{evtl.\@\xspace}
\newcommand{\ggf}{ggf.\@\xspace}
\newcommand{\bzgl}{bzgl.\@\xspace}
\newcommand{\so}{s.\nolinebreak[4]\hspace{0.125em}\nolinebreak[4]o.\@\xspace}
\newcommand{\iA}{i.\nolinebreak[4]\hspace{0.125em}\nolinebreak[4]A.\@\xspace}
\newcommand{\sa}{s.\nolinebreak[4]\hspace{0.125em}\nolinebreak[4]a.\@\xspace}
\newcommand{\su}{s.\nolinebreak[4]\hspace{0.125em}\nolinebreak[4]u.\@\xspace}
\newcommand{\ua}{u.\nolinebreak[4]\hspace{0.125em}\nolinebreak[4]a.\@\xspace}
\newcommand{\og}{o.\nolinebreak[4]\hspace{0.125em}\nolinebreak[4]g.\@\xspace}
\newcommand{\oBdA}{o.\nolinebreak[4]\hspace{0.125em}\nolinebreak[4]B.\nolinebreak[4]\hspace{0.125em}d.\nolinebreak[4]\hspace{0.125em}A.\@\xspace}
\newcommand{\OBdA}{O.\nolinebreak[4]\hspace{0.125em}\nolinebreak[4]B.\nolinebreak[4]\hspace{0.125em}d.\nolinebreak[4]\hspace{0.125em}A.\@\xspace}
\newcommand{\norm}[1]{\left\lVert#1\right\rVert}
\newcommand{\abs}[1]{\lvert#1\rvert}
\newcommand{\independent}{\protect\mathpalette{\protect\independenT}{\perp}}
\def\independenT#1#2{\mathrel{\rlap{$#1#2$}\mkern2mu{#1#2}}}
\newcommand{\fig}{Figure\@\xspace}
\newtcolorbox[auto counter, number within=section, number freestyle={\noexpand\thechapter.\noexpand\arabic{\tcbcounter}}]{threm}[2][]
{
  colframe = tuorange!40,
  colback  = tuorange!20,
  coltitle = tuorange!20!black,  
  title    = Theorem~\thetcbcounter: {#2},
  #1,
}
\newtcolorbox[auto counter, number within=section, number freestyle={\noexpand\thechapter.\noexpand\arabic{\tcbcounter}}]{example}[2][]
{
  colframe = tugray!40,
  colback  = tugray!20,
  coltitle = tugray!20!black,  
  title    = Example~\thetcbcounter: {#2},
  #1,
}

\newtcolorbox[auto counter, number within=section, number freestyle={\noexpand\thechapter.\noexpand\arabic{\tcbcounter}}]{proof}[2][]
{
  colframe = tugreen!40,
  colback  = tugreen!20,
  coltitle = tugreen!20!black,  
  title    = Proof~\thetcbcounter: {#2},
  #1,
}

\newtcolorbox[auto counter, number within=section, number freestyle={\noexpand\thechapter.\noexpand\arabic{\tcbcounter}}]{definition}[2][]
{
  colframe = tucitron!40,
  colback  = tucitron!20,
  coltitle = tucitron!20!black,  
  title    = Definition~\thetcbcounter: {#2},
  #1,
}

\newtcolorbox[auto counter, number within=section, number freestyle={\noexpand\thechapter.\noexpand\arabic{\tcbcounter}}]{algo}[2][]
{
  colframe = tugraybg!40,
  colback  = tugraybg!20,
  coltitle = tugraybg!20!black,  
  title    = Algorithm~\thetcbcounter: {#2},
  #1,
}

\let\vec\mathbf
\newcommand{\vect}[1]{\boldsymbol{#1}}
% Leere Seite ohne Seitennummer, naechste Seite rechts
\newcommand{\blankpage}{
 \clearpage{\pagestyle{empty}\cleardoublepage}
}
% Keine einzelnen Zeilen beim Anfang eines Abschnitts (Schusterjungen)
\clubpenalty = 10000
% Keine einzelnen Zeilen am Ende eines Abschnitts (Hurenkinder)
\widowpenalty = 10000 \displaywidowpenalty = 10000
% EOF

% color palette by Erich Schubert https://bitbucket.org/ls8-tudo/beamer-erich/src/template/
\xdefinecolor{tugreen}{RGB}{132, 184, 24} % source: official
\xdefinecolor{tudarkgreen}{RGB}{100, 150, 0} % source: official (purpose: web design)
\xdefinecolor{tulightgreen}{RGB}{224, 234, 204} % source: official (purpose: web design)
\xdefinecolor{tugray}{RGB}{207, 208, 210} % source: Powerpoint template diagram
\xdefinecolor{tugraybg}{RGB}{178, 179, 204} % source: official (purpose: web design)
\xdefinecolor{tuorange}{RGB}{227, 105, 19} % source: another template
\xdefinecolor{tuyellow}{RGB}{242, 189, 0} % source: another template
\xdefinecolor{tucitron}{RGB}{249, 219, 0} % source: another template
\xdefinecolor{tublue}{RGB}{46, 134, 171} % source: imaginary
\xdefinecolor{tudarkblue}{RGB}{30, 89, 114} % source: imaginary
\xdefinecolor{tuviolet}{RGB}{61, 52, 139} % source: imaginary
\xdefinecolor{tumagenta}{RGB}{139, 52, 88} % source: imaginary
\xdefinecolor{tudarkergreen}{RGB}{75, 98, 44} % source: Stefan Michaelis
\xdefinecolor{tuolive}{RGB}{83, 145, 45} % source: Stefan Michaelis
\xdefinecolor{tulime}{RGB}{215, 215, 0} % source: Stefan Michaelis
\xdefinecolor{tugraygreen}{RGB}{217,233,229} % source: Stefan Michaelis